\chapter{\abstractname}
In this thesis we propose a new relay setup for multimedia streaming that allows
for avoidance of userspace processing by utilizing BPF programs in the Linux kernel.
In a sample implementation we demonstrate the feasibility of this approach by
designing a relay that is capable of forwarding packets between a server-side 
and a client-side QUIC connection while still being able to do adaptive bitrate
streaming based on client congestion.
\\
We show that this approach saves processing time and reduces latency compared to
userspace processing with the relay still adhering to specifications of the QUIC
standard and the ``Media over QUIC'' (MoQ) draft.
One limitation that is not addressed in depth in this thesis is the need for a
de- and encryption hardware offload onto a SmartNIC to allow the BPF program to access
the packet payload without any restrictions.
Since the QUIC standard is still fairly new we are confident that a solution for a
potential hardware offload will be found in future research.