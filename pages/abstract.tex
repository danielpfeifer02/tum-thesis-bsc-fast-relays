\chapter{\abstractname}

Multimedia streaming constitutes a significant portion of internet traffic,
aiming to provide a seamless user experience with low latency and high quality, 
even amidst network congestion. 
Analyzing the packet path from media servers to clients reveals opportunities 
for latency reduction, particularly within relays where packets traverse the 
network stack twice—from the physical layer up to the application layer 
and back down.

This thesis explores leveraging low-level kernel features to bypass most of the network 
stack traversal by directly forwarding packets from ingress to egress using eBPF\@. 
This approach introduces the challenge of userspace unawareness of forwarded packets. 
To address this, we propose a delayed notification setup that informs userspace post-forwarding. 
We utilize eBPF maps to provide a low-latency communication channel between userspace 
and the eBPF program.

Our proposed method necessitates additional mechanisms for tasks such as congestion 
control and retransmissions. 
These aspects are discussed in detail, along with a prototype demonstrating the 
basic functionality and showcasing latency improvements of our approach compared 
to traditional userspace processing.

% Multimedia streaming makes up a significant portion of internet traffic.
% One main goal of multimedia streaming is to provide a seamless experience 
% for the user.
% The aim is low latency and high quality while still considering external factors
% like network congestion.
% When looking at the path a packet takes from a media server to a client,
% it becomes apparent that there are some parts where latency can be reduced.
% One such part is within a relay where packets pass through the network stack
% twice, once up to the application layer and once back down.
% We discuss how one can use low-level kernel features to avoid network stack
% traversal by directly forwarding packets from ingress to egress.
% We do so using eBPF which will allow us to do all the necessary processing.


% Such an approach introduces the problem that the userspace will be unaware 
% of the packets being forwarded.
% We propose a setup that allows notifying the userspace of forwarded packets.
% Our notfication setup will work in a delayed manner, meaning that the userspace
% will only find out about the forwarded packets after they have been processed 
% and forwarded already.
% This provides a reduction in latency while still allowing for a similar level of 
% control as before.
% Our setup, however, will require additional mechanisms for things like congestion 
% control or retransmissions.
% All those will be touched upon in this thesis.

% We show that this approach saves processing time and reduces latency compared to
% userspace processing with the relay still adhering to specifications of the QUIC
% standard and the ``Media over QUIC'' (MoQ) draft.
% One limitation that is not addressed in depth in this thesis is the need for a
% de- and encryption hardware offload onto a SmartNIC to allow the BPF program to access
% the packet payload without any restrictions.
% Since the QUIC standard is still fairly new we are confident that a solution for a
% potential hardware offload will be found in future research.