\section{Structure of this Thesis}\label{sec:structure_of_thesis}

In~\autoref{chap:background}, this thesis will provide an overview of used technologies
and related ideas.
Section~\ref{sec:quic_bg} will give an introduction to the QUIC protocol and its main features 
and~\autoref{sec:ebpf_bg} will provide an overview of eBPF technology together with features related to 
our approach.
Section~\ref{sec:moq_bg} will introduce the ``Media over QUIC'' (MoQ) transport protocol which will be used 
for our application-level relay setup.
After that~\autoref{sec:rt_and_adaptive_bitrate_streaming} will explain the ideas and challenges of 
real time streaming as well as adaptive bitrate streaming while~\autoref{sec:related_work} will 
mention some work related to the aforementioned topics.
What will follow in~\autoref{chap:fast_relays} is a detailed description of the setup that 
allowed us to improve relay performance. 
We will look at the adaptations to the used QUIC library in~\autoref{sec:quic_adaptions} as well as 
our eBPF setup in~\autoref{sec:ebpf_setup}. Besides those two, we will also look at some more specific details and 
challenges in the subsequent sections.
In~\autoref{chap:testing}, we will then provide a basic performance analysis of our setup to show current improvements and limitations.
Finally, we will conclude with a summary together with some ideas for future work in this field in~\autoref{chap:conclusion}.