\section{Structure of this Thesis}\label{sec:structure_of_thesis}

In chapter\nobreakspace\ref{chap:background} this thesis will provide some overview of used technologies
and related ideas.
Section~\ref{sec:quic_bg} will give an introduction to the QUIC protocol and its main features and 
section\nobreakspace\ref{sec:ebpf_bg} will provide an overview of eBPF technology together with features related to 
our approach.
Section~\ref{sec:moq_bg} will introduce the `Media over QUIC' (MoQ) transport protocol which will be used 
for our application level relay setup.
After that section\nobreakspace\ref{sec:adaptive_bitrate_streaming} will explain the ideas and challenges of 
adaptive bitrate streaming while section\nobreakspace\ref{sec:related_work} will mention some work related to 
the aforementioned topics.
What will follow in chapter\nobreakspace\ref{chap:fast_relays} is a detailed description of the setup that 
allowed us to improve relay performance. 
We will look at the adaptations to the used QUIC library in section\nobreakspace\ref{sec:quic_adaptions} as well as 
our eBPF setup in~\ref{sec:ebpf_setup}. Besides those two we will also look at some more specific details and 
challenges in the subsequent sections.
In chapter\nobreakspace\ref{chap:testing} we will then provide a basic performance analysis of our setup to show current improvements and limitations.
Finally we will conclude with a summary together with some ideas for future work in this field in chapters\nobreakspace\ref{chap:future_work} 
and\nobreakspace\ref{chap:conclusion}.