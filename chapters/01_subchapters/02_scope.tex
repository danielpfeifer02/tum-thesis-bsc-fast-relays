\section{Scope}\label{sec:scope}

The main improvement this thesis aims to achieve is shortening the critical path a packet takes from a media server to a client.
This will be done by avoiding the immediate need of a packet traversing up the network stack to the application layer.
Instead any communication with the application layer will happen in a delayed fashion (after the packet was sent) by utilizing 
eBPF maps for storing any necessary (meta-)~information.
\\
This approach is highly dependent on the used standards and protocols.
This thesis will operate on top of the QUIC protocol~\parencite{rfc-9000} and the `Media over QUIC' (MoQ) 
transport protocol~\parencite{draft-moqtransport}.
For the application layer the quic-go library~\parencite{quic-go-repo} will provide the implementation and 
any additional programs will also be written in Go.
Since the setup is dependent on retreiving data from eBPF maps the QUIC library providing the implementation 
will need some adaptations.
We will mainly introduce simple function pointer style additions that allow the adapted library to be run 
both with and without the eBPF setup.
The developer of the relay will then also have more freedom to setup the eBPF part of the relay as they see fit
since the Go code that will interact with eBPF parts will also have be provided by said developer.
\\
We will also run a performance analysis on our implementation of the relay to confirm the potential this 
approach has.
These performance tests will look at the raw delay speedup as well as the impact on CPU utilization this 
setup has.
All the tests will be done in a lab-like environment to isolate the performance changes as best as possible
form any outside noise.
The payloads used will only contain dummy data since our approach does not interfer with payload contents 
and there is no need for creating and using real media stream data.
\\
Despite our approach only considering QUIC and MoQ, the general idea of the shown setup is not dependent on 
any of these protocols and can be changed to fit ones needs.
The only assumption that was made is that it is possible for the protocol to store meta-information like a packet
priority in the packet itself.
In our case we utilized one byte of the QUIC Connection-ID as a proof of conecept but in later implementations 
this could be changed to using any separately defined header that allows for the storage of such information.
Since we assume full knowledge over the eBPF setup when the payload is created at the server we can also 
assume the possibility to add meta-information at some pre-defined location in the packet.