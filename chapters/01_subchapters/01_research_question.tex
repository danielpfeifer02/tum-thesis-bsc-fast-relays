\section{Research Question}\label{sec:research_question}

As already mentioned above the usage of application specific approaches in networking allows for increased packet processing speed.
In this thesis we will consider a media streaming scenario that runs on top of QUIC by using the ``Media over QUIC'' (MoQ) transport protocol
~\parencite{draft-moqtransport}.
By using eBPF technology together with kernel hook points provided by the Linux-Kernel, we will try to find a setup that improves relay 
performance using eBPF programs that handle basic relay capabilities, such as packet forwarding and congestion control.
Since the QUIC protocol settles both in kernel and userspace we look into possibilities of delaying any userspace processing until \textbf{after} the
packet has been forwarded to the client.
This way the raw delay that the packet experiences from the initial media server to the client could be reduced. 
However, since QUIC is a connection oriented protocol, additional processing in the eBPF program is needed to ensure that a packet stays coherent 
with the state that the client-side of the QUIC connection expects.
We will investigate which additional processing steps are needed in our case, how they compare to challenges when expanding our approach to other
protocols and how they can be implemented in an eBPF program.
