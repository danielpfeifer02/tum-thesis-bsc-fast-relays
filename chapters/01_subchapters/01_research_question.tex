\section{Research Question}\label{sec:research_question}

As already mentioned above the usage of application specific approaches in networking allows for increased packet processing speed.
In this thesis we will consider a media streaming scenario that runs on top of QUIC by using the `Media over QUIC' (MoQ) transport protocol
~\parencite{draft-moqtransport}.
By using eBPF technology together with kernel hook points provided by the Linux-Kernel, we will explore the possibility of having an eBPF 
program that handles basic relay capabilities, such as packet forwarding and congestion control.
Since the QUIC protocol settles both in kernel and userspace we propose a setup that delays any userspace processing until \textbf{after} the
packet has been forwarded to the client.
This way the raw delay that the packet experiences from the initial media-server to the client is minimized, however, since QUIC is a connection 
oriented protocol, some additional processing in the eBPF program is needed to ensure that the packets are coherent with the state that the 
client-side of the QUIC connection expects.
Besides designing the setup of a relay that supports such an eBPF speedup, we will also provide a basic example implementation and evaluate its
performance considering packet delay improvement and CPU utilization.