\section{Setups}\label{sec:setups}

We test the performance of our prototype in a lab-like environment that 
allows us to maintain a stable network state.
We do this by using Linux network namespaces where each namespace represents
a different participant in the communication, i.e.~the server, the relay, and the client.
A schematic overview of the setup is shown in \autoref{fig:namespace-setup}.

% \vspace{0.5cm}
% \begin{figure}[H]
% \centering
% \begin{myverbatim}
        
%                                     interface: veth1                      interface: veth3
%                                     ip: 192.168.1.2/24                    ip: 192.168.1.4/24
%               interface: veth0       \              interface: veth2       \
%               ip: 192.168.1.1/24      \             ip: 192.168.1.3/24      \
%  __________  /                         \  _______  /                         \  __________ 
% |          |/                           \|       |/                           \|          |
% |  Server  |-------> veth0@veth1 ------->| Relay |-------> veth2@veth3 ------->|  Client  |
% |__________|<------- veth1@veth0 <-------|_______|<------- veth3@veth2 <-------|__________|
        
% \end{myverbatim}
% \caption{Logical setup including interfaces and IPs}\label{fig:logical-setup}
% \end{figure}
% \vspace{0.5cm}

\vspace{0.5cm}
\begin{figure}[H]
\centering
\begin{myverbatim}
 ______________________         _______________________________________        ______________________
|   Server namespace   |       |            Relay namespace            |      |   Client namespace   |
|     ____________     |       |    ______________   ______________    |      |     ____________     |
|    |192.168.10.1|    |       |   | 192.168.10.2 | | 192.168.11.2 |   |      |    |192.168.11.1|    |
|____|___veth0____|____|       |___|____veth1_____|_|____veth2_____|___|      |____|____veth3___|____|
            \                            /                 \                             /
             \                          /                   \                           /
              \                        /                     \                         /
               \                      /                       \                       /
                \ __________________ /_                      __\ ___________________ /
                /veth0-br|     |veth1-br\                   /veth2-br|      |veth3-br\
                |                    ___|                   |___                     |
                \_____v-net-0_______|NAT/                   \NAT|_________v-net-1____/
                        /              \                     /                \
               ip: 192.168.10.10        \                   /         ip: 192.168.11.10
               net: 192.168.10.0/24      \   ___________   /          net: 192.168.11.0/24
                                           /             \
                                          |   enp1s0f0    |
                                           \ ___________ /
                                                  |
                                                  |
                                              (INTERNET)

\end{myverbatim}
\caption{Local setup including bridges and namespaces.}\label{fig:namespace-setup}
\end{figure}
\vspace{0.5cm}

\subsection{Local Environment for Testing and Development}\label{subsec:namespace_environment}
The local environment we use for testing and developing allows us to setup the network in a way 
that fits the current needs of the prototype.
One example for such needs would be that we need a slight delay between sending a packet and 
receiving its ACK since the registering of a sent packet takes place after a packet 
is sent out. 
In a real-world scenario this would obviously also be the case, if the distances covered 
are large enough.

% \vspace{0.5cm}
% % \begin{lstlisting}
% \begin{myverbatim}
%     root@root:~# tc qdisc show dev veth1-br
%     qdisc netem 8001: root refcnt 17 limit 1000 delay 5ms
% \end{myverbatim}
% % \end{lstlisting}
% \vspace{0.5cm}

In our setup we introduce a queueing discipline (qdisc) to an interface
of the bridge that connects the relay and the client.
This qdisc introduces a delay of 5ms to each packet.
Using such qdiscs allows to simulate a network environment that is more realistic, 
e.g.~by introducing delays, like we did, or defining thresholds for packet loss.
In our setup, however, we only added a delay. % TODO: mention loss, delay 25% etc.?
% However, where a local setup turns out better for testing is its deterministic nature.
The fact that the delay of a bridge can be set to a fixed value allows us to better 
analyse and understand the impact of the different setups for example on packet jitter.
Also, we can be sure that delays, and potential improvements thereof, are not caused by external
network state but rather by the immediate forwarding our approach introduces.
That way we can be sure that changes in any observed delay are mainly due to setup changes 
we introduced to the relay.

\subsection{Physical Server Setup for Real-World Testing}\label{subsec:physical_server_setup}
% TODO: mention that the physical server posed problems because of an old kernel?
Our setup for testing is solely based on local namespace environments.
This suffices since we only look at delay reduction and CPU utilization, 
both of which only have a local effect that happens at the relay.
We left it open for future work to implement a real-world setup that 
potentially even uses multiple relays and observes their combined improvement
in a more sophisticated network setup.
This goes hand in hand with a working congestion-control extension to the eBPF and 
QUIC setup, that reacts to a real-world network environment with ubiquitous changes 
in congestion.
