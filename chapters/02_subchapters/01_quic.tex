\section{QUIC}\label{sec:quic_bg}
% TODO: cite ~\parencite{quic-explained} for general explanation of QUIC
% TODO: cite \parencite{facebook-quic-usage} for QUIC usage at Facebook
% TODO: cite \parencite{google-quic-usage} for QUIC usage at Google
% TODO: cite \parencite{article-quic-usage} for general QUIC usage
Many fundamental internet protocols still used today have been around for 
a very long time.
For example, the Transmission Control Protocol (TCP) has been used as the backbone
of the internet for more than 40 years.
It has been designed to be reliable and to provide a connection-oriented
way of transmitting data, but the modern environment of the internet with
needs like lower latency, better multiplexing, or improved security makes it 
hard for TCP to keep up.
Limitations in the design and resulting issues like head-of-line blocking
have raised demand for a newly designed protocol that can keep up with the
modern internet. % TODO: head of line blocking citation
All of these issues, paired with the want for a more flexible development cycle
led to new creations.
QUIC, which started off as the ``Quick UDP Internet Connections'' protocol and has 
since been standardized by the IETF, with QUIC now being its own trademark, is a 
transport layer protocol built on top of UDP that is designed to be reliable, 
cryptographically secure and more performant than TCP\@.
QUIC, partly because it operates both in user- and kernel-space, has been designed to allow for a 
more rapid deployment cycle than TCP\@.
Similar to TCP, it is a connection-based protocol that uses TLS for encryption~\parencite{quic-explained}.
Back in 2018, QUIC was already the default protocol for the Google Chrome browser, which,
at the time, made up 60\% of the web browser market~\parencite{google-quic-usage}.
A little over two years later, Facebook, now Meta, was using QUIC for more than 75\% of 
their internet traffic which led to improvements regarding
request errors, tail latency, and header size~\parencite{facebook-quic-usage}.
As of August 2024, QUIC already made up 8.1\% of all internet traffic % TODO: update number
with support from pretty much every major browser
~\parencite{internet-quic-usage, article-quic-usage}.
On another note, Cloudflare-Radar has reportet, that at point of writing this thesis, 
30\% of HTTP traffic is HTTP/3, which uses QUIC~\parencite{cloudflare-radar}.
With big players like Google, Meta, or Microsoft putting emphasis on
using QUIC to improve their services, this number will likely increase even further.

\subsection{Connections and Streams}
Since QUIC is a connection-based protocol, some initial overhead to establish a connection is needed.
However, the design incorporates features that aim for an efficient way of establishing 
connections, e.g.\ by using 0-RTT (zero round-trip-time) handshakes. 
Latency improvements like the 0-RTT handshake, however, come at the cost of security since that opens 
the door for replay attacks.
QUIC's 1-RTT handshake does not have this issue, while still being faster
than e.g.\ the handshake of TCP/TLS, since it handles all setup tasks using only
a single round trip.
Another part where QUIC tries to optimize connection management is the use of streams.
Streams are designed to be lightweight and can be opened without the need for a handshake.
This goes as far as one single packet being able to open a new stream, transferring stream data,
as well as closing the stream again.
This allows for new techniques to improve data transmission and will also be part of the fast-relay 
setup in this thesis.
Aside from streams it is also possible to send data via unreliable 
datagrams~\parencite{rfc-9221}.
This is possible since QUIC is based on UDP\@.
It further improves the versatility of the protocol and allows 
for new ways of optimizing data transmission.

\subsection{quic-go}
There are many implementations of the QUIC protocol available, providing libraries for a lot of 
today's most popular programming languages.
The implementation we settled on for this thesis is the quic-go library which provides a pure Go 
approach to implementing the standards of RFC-9000~\parencite{rfc-9000}, 
RFC-9221~\parencite{rfc-9221} as well as some others which are not 
important for our usecase. 
However, since we need some special behavior of the userspace part of QUIC, we will introduce some 
modifications into quic-go.  
Those modifications will be explained further in~\autoref{sec:quic_adaptions}.

\subsection{QUIC's Importance to Fast-Relays}
The QUIC protocol will be a fundamental part of the fast-relay setup in this thesis, yet the ideas used 
to make relays faster is not limited to QUIC\@.
QUIC is chosen as an example protocol due to its increasing popularity, which offers big potential 
in early adoption and deployment of fast-relays.
Also, the easy incorporation of changes into libraries providing RFC implementations makes it a good 
starting point for experimenting with what can and cannot be done regarding our research questions.
This includes the possibility of neglecting the difficulties that the heavy encryption of QUIC brings with 
it by just turning off the related functionality.
Another reason we opted for QUIC is that it allows for easy packet dropping.
In order to do that we just need to send each frame in a lightweight unidirectional stream and in 
case of a drop of said frame, we can just close the corresponding stream.
This would not be possible with TCP, due to its reliability-based nature.

% This includes the possibility of mitigating missing technologies, mainly for offloading QUIC % TODO: better wording?
% decryption and encryption onto hardware.
% To conquer that, the existing protocol libraries can be modified easily to simulate any needed behavior.