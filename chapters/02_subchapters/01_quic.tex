\section{QUIC}\label{sec:quic_bg}
% TODO: cite ~\parencite{quic-explained} for general explanation of QUIC
% TODO: cite \parencite{facebook-quic-usage} for QUIC usage at Facebook
% TODO: cite \parencite{google-quic-usage} for QUIC usage at Google
% TODO: cite \parencite{article-quic-usage} for general QUIC usage
The Transmission Control Protocol (TCP) has been used as the backbone
of the internet for more than 40 years.
It has been designed to be reliable and to provide a connection-oriented
way of transmitting data, but the modern environment of the internet with
its need for increasing throughput make it hard for TCP to keep up.
Limitations in the design and resulting issues like head-of-line blocking
have raised demand for a newly designed protocol that can keep up with the
modern internet. % TODO: head of line blocking citation
The `Quick UDP Internet Connections' protocol, short QUIC protocol, is a 
transport layer protocol built on top of UDP that is designed to be reliable, 
cryptographically secure and more performant than TCP\@.
It was intended to be the successor of TCP and it has its origins at Google before 
the standardization by the IETF began in 2016.
QUIC, partly because it operates both in user- and kernel-space, has been designed to allow for a 
more rapid deployment cycle than TCP\@.
Similar to TCP it is a connection-based protocol that uses TLS for encryption~\parencite{quic-explained}.
Already back in 2018, QUIC was the default protocol for the Google Chrome browser which,
at the time, made up 60\% of the web browser market~\parencite{google-quic-usage}.
A little over two years later, Facebook, now Meta, was using QUIC for more than 75\% of 
their internet traffic which led to improvements regarding
request errors, tail latency and header size~\parencite{facebook-quic-usage}.
As of May 2024, QUIC already made up 8.0\% of all internet traffic
with support from pretty much every major browser
~\parencite{internet-quic-usage}~\parencite{article-quic-usage}.
With big players like Google, Meta, Microsoft or YouTube putting emphasis on
using QUIC to improve their services, this number is likely to increase even further.

\subsection{Connections and Streams}
Since QUIC is a connection-based protocol, some initial overhead to establish a connection is needed.
However the design incorporates some features that aim for a more efficient way of establishing 
connections, e.g.\ by using 0-RTT (zero round trip time) handshakes.
Latency improvements like the 0-RTT handshake however come at the cost of security, since that opens 
the door for replay attacks.
Another part where QUIC tries to optimize connection management is the use of streams.
Streams are designed to be lightweight and can be opened without the need of a handshake.
It even goes as far that a single packet can contain the opening of a new stream, stream data
as well as the closing of the stream.
This allows for new techniques to improve data transmission and will also be part of the fast-relay 
setup in this thesis.
Aside from streams, apparent since QUIC is based on UDP, it is also possible to send data via
unreliable datagrams.
This further improves versatility of the protocol and allows for new ways of optimizing data transmission.

\subsection{quic-go and moqtransport}
The implementation of the proposed fast-relay setup will be based on the quic-go library, which provides a
pure Go implementation of the QUIC protocol as specified in the standards RFC-9000, RFC-9221 as well as some
others which are not that important in this thesis. Together with a modified version of the quic-go library,
the fast-relay implementation will also use the moqtransport library.
This library brings the `Media over QUIC' (MoQ) protocol to Go and will be used as a media transport protocol 
when looking at the impact of fast-relays on adaptive real-time video streaming.
The MoQ protocol is being standardized by the IETF since July 2023 and has yet to be finalized. 

\subsection{QUIC and Fast-Relays}
The QUIC protocol will be a fundamental part of the fast-relay setup in this thesis, yet the ideas used 
to make relays faster is not limited to QUIC and can be extended to other protocols as well.
QUIC is chosen as an example protocl due to its increasing popularity which offers big potential 
in early adoption and deployment of fast-relays.
Besides that, the existing implementations of QUIC related standards provide a good starting point for
an implementation, despite the difficulties that the heavy encryption of QUIC brings with it.
To mitigate missing technologies, mainly for offloading QUIC decryption and encryption onto hardware,
the existing protocol libraries can also be modified easily to simulate any needed behavior.
