\section{Related Work}\label{sec:related_work}

Our approach combines ideas from multiple different 
related publications.
This section will give a brief overview on some related work 
that uses similar concepts such as packet handling using a BPF 
program or bypassing the Linux network stack to improve performance.

\subsection{eQUIC Gateway}
There have been previous publications on making QUIC more efficient by using BPF programs
such as~\parencite{equic-gateway}, where a BPF program is used together with the Linux
eXpress Data Path (XDP) to filter packets based on information provided by the userspace.
This approach provided significant performance improvements with an increase of throughput
by almost a third and a reduction of CPU time consumption caused by filtering packets by
more than 25\%.
This shows that a setup leveraging Linux kernel features such as BPF has a lot of potential
to improve the current infrastructure.
% TODO: more specific info on how this paper's setup looks like / differs from ours?
% TODO: mention that XDP is not feasible for our setup because it has no egress hook?

\subsection{Kernel Bypass}
Another interesting approach that follows a similar idea of speeding up packet processing
by avoiding the Linux network stack is~\parencite{kernel-bypass-msc-thesis}.
The difference in this work is that DPDK is used to bypass the network stack to 
then process packets in userspace instead of using BPF programs like this thesis does.
This, for example, offers more flexibility as the userspace program is not as limited (e.g.\ 
by the eBPF verifier) as the eBPF program but might also lead to slightly more system calls,
especially in the setup of a system, when user- and kernel-space need to communicate.
% TODO: clear enough / everything covered and correct?

\subsection{Priority drop}
The idea of dropping packets based on their priority to adapt a connection
in a congestion event has also been around for a while.
~\parencite{media-streaming-prio-drop} explores this in more detail.
Mainly improvements like a more tailorable congestion handling than the sole usage
of discrete video quality levels as well as an improvement to, potentially 
randomized, frame dropping are discussed.
This thesis, similar to~\parencite{media-streaming-prio-drop}, will not focus
on how the priority for packets is determined but rather on how those marked
packets are handled.
For this, it is assumed that a higher level protocol has correctly determined 
the packet priorities and can handle the drop of packets with lower priority
in case of limited bandwidth.
% TODO: explain more on different ways of encoding priorities e.g. I- and 
% TODO: P-frames in video streaming?