\section{eBPF}\label{sec:ebpf_bg}
In 1992 a technology called `Berkeley Packet Filter' (BPF) was introduced into 
the Unix kernel.
By using BPF it is possible to attach a small BPF-program to some pre-defined hook points in 
the network stack of the kernel and filter packets there in a stateless manner.
This provided more efficiency since the packets did not need to be copied into 
userspace anymore but could directly be processed in the kernel.
One downside to such an approach however is that BPF-programs are limited by the 
so-called `BPF-verifier' which needs to check every BPF-program for safety e.g. 
to avoid infinite loops or access to invalid memory from withing kernel space. 
Today, the initial technology of BPF has evolved into `extended BPF' (eBPF) and 
allows for more versatile use cases.

\subsection{eBPF Hook Points}
TODO

\subsection{eBPF Verifier}
TODO

\subsection{Important eBPF Concepts}
TODO

\subsection{eBPF and Fast Relays}
TODO
