\section{Congestion Considerations}\label{sec:congestion_considerations}
QUIC, like many other modern transport protocols, contains congestion 
control mechanisms regulating the rate at which data is sent to a client.
This is primarily done to avoid the network becoming congested, but it also
has the secondary effect of circumventing the problem of overwhelming the client.
Simply forwarding all packets the relay receives from the server would cause 
the relay-client connection to no longer have its own congestion control.
Rather the rate at which the relay sends/forwards to the client would be determined
by the server's congestion control algorithm, i.e.~the network congestion between 
server and relay.
Obviously, this is not a desirable situation, so our approach suggests the eBPF 
program at egress to have its own congestion control functionality.

\subsection{Client Congestion}
Already hinted at in figure~\ref{fig:forward-registration}, it is shown that once a packet is 
registered, there will also be a map update that will be triggered by the congestion controller.
This map update will tell the eBPF egress program how much data it is allowed to send out. 
In figure~\ref{fig:forward-registration} this is visualized exemplary as ``Bytes per Second Limit''
but the idea is that both the function determining how limits and thresholds are calculated from 
the information on incoming packets as well as the actual handling within the egress program 
will be application-specific and defined by the relay engineer.
We experimented with approaches that use the QUIC internal measurements like the RTT, 
introduce new measurements like exponential-weighted moving averages, or even use an 
out-of-band connection where the relay expects direct feedback from the client.
All these possibilities show that there is a lot of room for experimentation and optimization
in this area.
This, however, will not be explored further in this thesis and is left for future work.
\\

\subsection{Packet Filtering and Dropping} % TODO: not sure if like this part
Assuming that the network congestion state is known and communicated to the relay, one can 
use the priority-information within a packet (that is expected to be set by the server) to
decide which packets should be forwarded and which ones should be dropped.
One difficulty in this approach is that the dropping mechanism essentially works as an online-
algorithm, meaning that it does not have full knowledge of the traffic, especially not of 
future packets.
This means that a case like the following could happen:
\begin{itemize}
    \item The traffic contains packets within the priority range of 1 to 5 (5 being the highest priority).
    \item Given the current network congestion to the client, the relay decides to drop 
            all packets below priority 3 as well as 50\% of the packets with priority 3.
    \item The remaining byte limit to be sent out is running low, and a packet with priority 3 
            comes in which is sent since the relay already dropped a lot of previous priority 3 packets.
    \item The next packet turns out to be a priority 5 packet which would overshoot the byte limit if sent.
\end{itemize}
In this example, one could use many different heuristics to handle the situation.
One could always keep enough byte limit left so that a high-priority packet can always be sent.
This, however, essentially just lowers the byte limit for all other packets while making it higher 
for high-priority packets.
Therefore an individual byte limit per priority could also be used right away.
Also, this approach might cause problems if a lot of high-priority packets come in at once, 
e.g.~in case of very ``bursty'' traffic.
Another way that could be used to handle this uncertain situation is to allow for temporary 
overflows of the byte limit.
This would make the limit more of a soft limit that can be exceeded for a short time.
However, this is another heuristic highly dependent on the specific use case and
traffic patterns, which is why we did not implement it in our prototype.
Overall we can say that implementation-wise it is not hard to drop packets, but the actual
difficulty lies in finding a reasonable way of deciding which packets to drop.