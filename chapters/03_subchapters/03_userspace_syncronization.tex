\section{Userspace Synchronization}\label{sec:userspace_synchronization}

Since one of the main ideas we propose is to avoid passing a packet all the way
through the network stack up to userspace at the relay, we create the problem
that the application itself is not aware of packets that are being sent.
In order to conquer this issue, we suggest a setup that establishes communication
between the application- and the eBPF-program.
To keep the improvement we gained by not passing packets to userspace during the 
critical path, this communication will happen in a delayed fashion, that is 
decoupled from the actual sending of media data.
\\
The main resource we use for this communication will again be eBPF maps that
will contain information on the time a packet was sent together with protocol-specific 
data like packet numbers or stream identifiers (for both packet number 
and stream id the old and the new, i.e.~tranlsated, values will be stored).


% \subsection{User Space Avoidance}
% TODO 

\subsection{Subscription and State Management}
As already mentioned in setion~\ref{sec:ebpf_setup}, an eBPF program handling incoming
traffic from the client will save client connection information like MAC address, IP 
address, and so on in a map for later access.
Also, an internal counter will give each client a unique identifier. % TODO: mention somewhere that this identifier has to be sequential and therefore does not support "unsubscribing" yet
With that, the only thing that happens in terms of communication between the application 
and the relay in case of a new subscription is that the application will update the number 
of clients counter that is accessed by the eBPF program and used for packet duplication purposes.
\\
Regarding stream state management, there is also little need for communication since the 
server is expected to use QUIC's unidirectional streams for sending the media data. 
That means the relay does not really need to know about the stream details except in case it 
has to trigger a retransmission.
If that is the case, the stream id contained in the packet (meta-)data that was read from the eBPF map 
will be used to manually create a stream with the correct id.
It is important to manually set the correct id since the relay might not use the same id for the 
next unidirectional stream it opens.
Also, the client expects the retransmission to be sent on the same stream id as the original packet
since retransmission happen within the same stream-context.

\subsection{Relay Caching}
Regarding the caching of data within a relay, which is required by the MoQ standard, we handled this 
by passing on an unaltered copy of any incoming packet from the server to the application at the same 
time we forward all the other packet copies to egress.
Now the application is able to receive any data from the server as if this was a normal connection and 
store, e.g.~the last second(s) worth of data in a cache.
Then the relay could, once a new connection is established, parallel to incrementing the kernel counter 
representing the number of clients also send out cached data already so that the client receives it
as early as possible.
\\
One aspect of such a setup that we still left open is the point in time where the relay should stop 
sending cached data since the forwarded data is up-to-date.
This also includes the question of how the client can handle potentially duplicate media data if the 
cached- and the forwarded data happen to overlap.
Such questions would likely require some further experimenting and testing to find a good solution.
