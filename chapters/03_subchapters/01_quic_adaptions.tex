\section{QUIC Adaptions}\label{sec:quic_adaptions}
As was already mentioned in the previous chapter, our setup requires some adaptations
to the quic-go library.
One initial change that was necessary was to turn off packet en- and decryption, 
happening within quic-go.
Given that we operate on the QUIC-header data within the eBPF-program, we need access 
to fields that are encrypted using QUIC header-protection.
For obvious reasons sending unencrypted packets is not something that would be wanted in 
a production environment, but for our setup, it is required since there was no fitting hardware offload 
was available at the time of writing that would have allowed us to `push down' en- and decryption 
onto a smartNIC\@.
Given that such a hardware offload is added in the future, the en- and decryption can be
turned on again, which makes this change more of a temporary solution to show the feasibility
of our approach.

\subsection{Function-Pointer Style Additions}
Another type of change that we needed to introduce into the quic-go library is caused by 
connection state management.
We essentially added support for communication with the eBPF-program by using an 
approach similar to C-style function pointers.
\\
On multiple locations, we added conditional function calls like the one depicted in 
\autoref{changes:function-pointer}.
The function that is called here will be defined by the developer of the relay and 
therefore allow for customizability without the need for changing the library itself.

\vspace{0.5cm}
\noindent\begin{minipage}{\textwidth}
\begin{lstlisting}[style=GoStyle, label=changes:function-pointer, caption=An example of a function-pointer addition to the quic-go library.]
    /* Function pointer call within actual quic-go code */
    if packet_setting.ConnectionUpdateBPFHandler != nil /* && potentially other conditions */ {
	    packet_setting.ConnectionUpdateBPFHandler(connId.Bytes(), uint8(connId.Len()), p.connection)
	}
\end{lstlisting}
\end{minipage}

\vspace{0.5cm}
\noindent\begin{minipage}{\textwidth}
\begin{lstlisting}[style=GoStyle, label=changes:signature-function-pointer, caption=Only the signature will be defined within the library itself.]
    /* Function pointer signature definition within additional config file */
	ConnectionUpdateBPFHandler      func(id []byte, l uint8, conn QuicConnection) = nil
\end{lstlisting}
\end{minipage}

The definition of the function that the developer of the relay wished to be executed at the specifically
defined points will be defined locally in the relay code and provided to the configuration of the quic-go library.
An example of how this could look like is shown in \autoref{changes:definition-function-pointer}.

\vspace{0.5cm}
\noindent\begin{minipage}{\textwidth}
\begin{lstlisting}[style=GoStyle, label=changes:definition-function-pointer, caption=An example of how the addition looks on the relay side.]
    /* Definition of the function within the local relay code */
    func UpdateConnectionId(id []byte, l uint8, conn packet_setting.QuicConnection) {
        /* handle the connection update by interacting with the eBPF-program */
    }   

    /* Providing the function to the quic-go library */
    func main() {
        /* ... */
        packet_setting.ConnectionUpdateBPFHandler = common.UpdateConnectionId
        /* ... */
    }
\end{lstlisting}
\end{minipage}

The need for these additions arises since the eBPF-program works with its own copy of the current state of a connection.
This, for example, includes the connection-id that will be used when changing the packet header before sending it out.
Since a connection-id can change, i.e.~be updated or retired, during the lifetime of a connection, we need a way to inform 
the eBPF-program to no longer use outdated state-information.
These function-pointer style additions provide a minimal way of adding such functionality without limiting flexibility 
or adding too much application-specific code to the library itself, as it would be the case if the library would access 
the eBPF-Maps directly.

% TODO: mention changes which are not function-pointer style
\subsection{Direct Changes to the Library}

\iffalse
% open stream with priority (+ datagram)

% turn off crypto (reaction to CRYPTO_TURNED_OFF flag)

% connection id retirement specific stuff (switch to priority id)

% fixed sizes for fields e.g. conn id, stream id, etc.(just for ease of development)

% e.g. only single stream frame inside packet (for ease of development) since general approach too complex for verifier?

% retransmission functions that open new stream with correct id

registerBPFPacket function for connection

% prio enum. (prolly could be also given by relay dev) % TODO: not interessting enough i'd say
\fi

Beside the simple function-pointer style additions, we also had to make some direct changes to the library.
These include simplifications of the packet structure to make the implementation of a prototype easier 
but also changes that are necessary for the whole approach to work.
The necessary state changes mainly required internal state adjustments that would not be possible from outside 
the library because of missing access / visibility.
The optional turn-off for the packet en- and decryption based on the value of the \verb|CRYPTO_TURNED_OFF| flag
also required some direct changes to the library but will not be mentioned further since this is not a 
focus of this chapter.

\subsubsection*{Simplifications of Packet Structure}
As already mentioned we added some changes to the quic-go library to make a prototype implementation easier.
The first one is that we fixed the sizes of some variable length fields like the connection-id or the stream-id.
This was mainly to avoid the need to figure out the correct sizes within the eBPF-program as this would have 
resulted in approaches that would have to be very carefully turned into eBPF-compatible code due to verifier
restrictions.
Besides fixing the length of some fields we also limit the number of frames per packet to one.
Normally a QUIC packet can contain multiple frames, especially stream-frames, but this would have required 
a packet traversal within the eBPF-program that, again, would have been harder to implement considering all
verification constrictions.

\subsubsection*{Stream Priorities}
The first change that is not for simplification of our prototype implementation, but actually needed for 
our approach to work is the addition of opening a stream with a specific priority.
Our assumptions define that the server is the one marking the packets with the correct priority, which in 
our case means sending them via the correct stream.
Such a stream is bound to a specific priority and our code changes will make sure that the connection-id
that is used for any packet sent on this stream will contain the correct priority-id.
Figure~\ref{fig:priority-stream} visualizes the setup as well as our rudimentary approach of saving the 
priority value as the first byte of the connection-id.

\begin{figure}[H]
    \centering
    TODO
    \caption[Opening a stream with a specific priority]{Opening a stream with a specific priority.
    The server opens streams with specific priorities, which are then used to send packets with the 
    corresponding priority.}\label{fig:priority-stream}
\end{figure}

This approach of having the priority value saved within the connection-id caused the need for another 
change in the library.
Due to the periodic retirement of connection-ids, we need to make sure that at any point in time there 
exists one connection-id per connection for each priority.
Otherwise we might not be able to mark a packet with the correct priority.
This led us to introduce some additional logic to the code executing the retirement of connection-ids.
There we will make sure that before a connection-id is retired, either one with the same priority value
already exists or is created.
That solves the problem of not being able to mark a packet with the correct priority.

\subsubsection*{Retransmission}
Another direct change that is needed is that of a specific \verb|OnLost()| function for packets that 
have been forwarded by the eBPF-program.
Since the relay state is not necessarily aligned with the server state, and the relay does not know
about the stream a packet was sent on, it is not possible to reuse the plain \verb|OnLost()| function
of the quic-go library.
Instead our new function needs to lookup the stream-id of the lost packet and open a new stream with
that same id to ensure that the client correctly receives the retransmission.
The relay also needs to tell the underlying eBPF-program that a packet is part of a retransmission and 
that, even if a stream was actually created by the relay, the egress program should treat it like it was
created by the server.
That last part is necessary for the stream-id translation of the egress program, which is explained further
in section\nobreakspace\ref{sec:client-egress}, to work correctly.
This functionality of informing the eBPF program about the retransmission, however, is again realised by a 
function-pointer style addition.

\subsubsection*{Visible Endpoint for Packet Registration}
The last direct change that we added was the introduction of an additional function \verb|RegisterBPFPacket| 
on a quic-go connection object that allows the relay to register a packet.
Since the packet registration requires access to internal state of the connection this also needed to be 
done as an actual change to the library.
Now the relay can just read the necessary information of a packet that needs to be registered
from the eBPF-maps and then pass it on to this function which will handle the registration.
This also provides a good separation between the Go code that handles eBPF communication and the actual
QUIC connection handling.