\section{Integration and Prototype}\label{sec:integration_and_prototype}

In order to show the feasibility of our approach, we have built a prototype 
that implements the suggested setup.
It is capable of streaming an example video as well as mock-payload data from 
a server to a client via a relay.
This setup will be open source and available on GitHub.
Some of the mentioned functionalities like retransmission, caching and adaptive
bitrate streaming are not fully implemented yet but should not require major
additions since the infrastructure already considers them.
The absence of these functionalities will not have a drastic impact on further 
testing however since the retransmission and the adaptive bitrate streaming will not
be needed in out lab-environment and the caching is not expected to have a big impact
either.

\subsection{Compatibility}
The application layer of the prototype is written in Go but this is not a requirement
itself.
The eBPF program expects QUIC but could also be changed without too much effort to
support any other underlying transport protocol.

\subsection{Source Code Repositories}\label{sec:source_code_repos}
For the development of the relay and the eBPF programs, we have come up with the following repositories:
\begin{itemize}

    \item \textbf{Fast-Relay}~\parencite{adaptive-moq-repo}:
    This is the main repository providing the eBPF program implementations as well as examples of 
    server, relay and client implementations in Go.
    
    \item \textbf{Quic-Go Adaptation}~\parencite{quic-go-prio-packs-repo}:
    This repository is a fork of the QUIC library ``quic-go''~\parencite{quic-go-repo} and provides a 
    plain Go implementation of the QUIC protocol.
    For our thesis we needed to make some adaptations to the library to support some hook points for 
    separate functions which should be specifically designed to handle the underlying eBPF setup with its eBPF-Map usage. 
    
    \item \textbf{MoQ-Transport Adaptation}~\parencite{priority-moqtransport-repo}:
    This repository is a fork of the ``MoQ-Transport''~\parencite{draft-moqtransport} protocol repository and provides
    some needed adaptations to our examples. One such adaptation is that the server needs to support a 
    categorization of payloads into different priorities in order for the eBPF program to be able to 
    deliberalty drop packets in case of congestion.
    Getting these priorities could be as simple as differentiating only between I- and P-frames in a video 
    stream or more complex based on the needs of the application and the wanted granularity of the congestion 
    control.
    
\end{itemize}