\section{Future Work}\label{sec:future_work}

Here we will mention some parts that still require additional work 
to be done.
The areas that will be mentioned will provide a good starting point
in case one wants to build upon or extend our work.

\subsection{Hardware Offload}
This thesis heavily relies on the fact that the relay can
access certain fields (e.g.~packet-numbers) of the packet, 
which are generally not accessible without prior decryption.
In the current setup, this is made possible by turning off 
encryption altogether but to be of any use in a real-world
scenario, the encryption of incoming and the decryption of
outgoing packets would need to be pushed down below the lowest 
used BPF hook point in the stack.

Future work could focus on developing a hardware offload 
setup similar to those already available for TCP/IP checksum 
offloading~\parencite{tcp-ip-offload-engine}.
We looked for possible smartNICs and libraries that would allow us
to offload en- and decryption.
However, at the time of writing, we could not come up with any viable solution
that works with our current setup. 
We believe one could possibly even put the eBPF program itself on the smartNIC as 
another approach suggests~\parencite{ebpf-offload-smartnics}.
The cited work only considers eBPF/XDP offloading, so further 
research would be required to see if this could be adapted to eBPF/TC 
programs as well.

Despite the current lack of options, previous work in this direction has already been 
done since at least 2019~\parencite{quic-nic-offload}, which makes us hopeful that
integrating smartNIC support into our prototype can be achieved in the near future.

\subsection{Compatibility Expansion}\label{sec:compatibility_expansion}
This thesis used the QUIC protocol together with MoQ to demonstrate how fast-relays, 
which circumvent userspace by utilizing eBPF programs, can be designed.
However, generally speaking, the design of fast-relays is not limited to
any of these protocols and could be expanded, given modifications to 
necessary fields are possible (i.e.~not prevented by encryption),
the protocol provides a possibility to drop packets (i.e.~no TCP, 
but something similar to QUIC),
and there is a way to accommodate the priority of a packet in the packet itself.
The final point could always be realized by using part of the payload
which forces a deeper packet inspection within the BPF program but 
avoids the need to fit the priority into the header of an existing 
protocol.

Expanding our approach to other protocols could be interesting for 
many areas outside of media streaming, examples being gaming, 
telesurgery or financial services where a delay-decrease of 
microseconds could help make a system more deterministic and 
thus more reliable.

\subsection{Prototype Completion}\label{sec:prototype_completion}
As mentioned already during the setup description, our prototype 
does leave out some parts of the proposed design.
Despite them not being essential for our proof of concept,
both in functionality and performance, they are still important
for a final product that is intended to be used in a real-world
scenario.

Future work could focus on finalizing the prototype by researching 
and implementing best practices for the areas of congestion handling
and retransmission management, among others.
A fully functional prototype could then be tested in a real network
environment to see how common network conditions, that were not considered
within our namespace environment, affect the performance of the system.
