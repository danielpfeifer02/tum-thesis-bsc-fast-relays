\section{Conclusion}\label{sec:conclusion}
In this thesis, we have shown how one can improve the performance of a relay in a media streaming scenario 
by using eBPF technology.
We explained needed high-level concepts and provided insights into implementation details of a prototype 
showing the feasibility of our approach.
Concluding, we can say that leveraging the high-performance capabilities of eBPF programs and forfeiting 
some universality of a relay-design can lead to faster and more deterministic packet processing with 
a similar CPU load.
We thereby answered the initial research questions regarding the possibility of a performance improvement 
and the avoidance of userspace processing.
Our prototype showed the needed communication between the userspace and the eBPF program to keep coherency and, by doing so, answered another research subquestion regarding the necessary communication.
Lastly, our final hints on expanding our setup can be seen as a starting point for future work and answering the last of our initial questions regarding generalization.
