\section{Hardware Offload}\label{sec:hardware_offload}
This thesis heavily relies on the fact that the relay can
access certain fields (e.g.~packet numbers) of the packet, 
which are generally not accessible without prior decryption.
In the current setup, this is made possible by turning off 
encryption altogether but to be of any use in a real-world
scenario, the encryption of incoming and the decryption of
outgoing packets would need to be pushed down below the lowest 
used BPF hook point in the stack.
\\
Future work could focus on developing a hardware offload 
setup similar to those already available for TCP/IP checksum 
offloading~\parencite{tcp-ip-offload-engine}.
We looked for possible smartNICs and libraries that would allow us
to offload en- and decryption.
However, at the time of writing, we could not come up with any viable solution
that works with our current setup. 
We believe one could possibly even put the eBPF program itself on the smartNIC as 
another approach suggests~\parencite{ebpf-offload-smartnics}.
The cited work does consider eBPF/XDP offloading, so further 
research would be required to see if this could be adapted to eBPF/TC 
programs as well.
\\
Despite the current lack of options, previous work in this direction has already been 
done since at least 2019~\parencite{quic-nic-offload}, which makes us hopeful that
integrating smartNIC support into our prototype can be achieved in the near future.