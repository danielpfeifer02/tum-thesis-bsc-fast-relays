\section{Hardware Offload}\label{sec:hardware_offload}
This thesis heavily relies on the fact that the relay can
access certain fields (e.g.~packet numbers) of the packet, 
which are generally not accessible without prior decryption.
In the current setup, this is made possible by turning off 
encryption alltogether but to be of any use in a real-world
scenario, the encryption of incoming and the decryption of
outgoing packets would need to be pushed down below the lowest 
used BPF hook point in the stack.
This means that a hardware offload of encryption and decryption
similar to what is done for TCP/IP checksums would be necessary.
% TODO: cite    https://de.wikipedia.org/wiki/TCP_Offload_Engine or similar

Once compatible SmartNIC offload implementations are available one can, 
besides en- and decryption, also offload the BPF program itself.
This then would provide another way of accelerating performance.
% TODO: cite    https://storage.corigine.com.cn/UploadFiles/file/2022-01-24/100418388672292.pdf

Some previous work in this direction has already been done since 
at least 2019~\parencite{quic-nic-offload} but for the purpose of 
this thesis we did not find any suitable open-source implementation 
that would allow incorporation into our fast-relay example 
implementation.