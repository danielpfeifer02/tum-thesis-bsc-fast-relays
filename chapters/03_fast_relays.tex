% !TeX root = ../main.tex
% Add the above to each chapter to make compiling the PDF easier in some editors.

\chapter{Fast-Relays}\label{chap:fast_relays}

In this chapter, we will dive into the specifics of our proposed fast-relay setup.
We will cover necessary library adaptations as well as setup specifics like eBPF programs 
and userspace synchronization.
\autoref{fig:route-layering} shows a high-level overview of a basic 
server-relay-client setup with its respective networking layers.
A conventional setup would have the packet travel into userspace at the relay, 
while our setup aims to reduce the critical path, as indicated by the red arrows.

\vspace{0.5cm}
\begin{figure}[htbp] % TODO: where to put this?
    \centering
    \includegraphics[width=0.7\textwidth]{figures/03_fast_relays/route-layering.drawio.pdf}
    \caption[Packet path schematic regarding network stack]{Conventional networking layers a packet passes.
    The red loop indicates again the ``short-cut'' that is utilized by the fast-relay 
    (eBPF packet-forwarding {-} no need to go up to userspace).}\label{fig:route-layering}
\end{figure}

\section{QUIC Adaptions}\label{sec:quic_adaptions}
As was already mentioned in the previous chapter, our setup requires some adaptations
to the quic-go library.
One initial change that was necessary was to turn off packet en- and decryption, 
happening within quic-go.
Given that we operate on the QUIC-header data within the eBPF-program, we need access 
to fields that are encrypted using QUIC header-protection.
For obvious reasons sending unencrypted packets is not something that would be wanted in 
a production environment, but for our setup, it is required since there was no fitting hardware offload 
was available at the time of writing that would have allowed us to `push down' en- and decryption 
onto a smartNIC\@.
Given that such a hardware offload is added in the future, the en- and decryption can be
turned on again, which makes this change more of a temporary solution to show the feasibility
of our approach.

\subsection{Function-Pointer Style Additions}
Another type of change that we needed to introduce into the quic-go library is caused by 
connection state management.
We essentially added support for communication with the eBPF-program by using an 
approach similar to C-style function pointers.
\\
On multiple locations, we added conditional function calls like the one depicted in 
\autoref{changes:function-pointer}.
The function that is called here will be defined by the developer of the relay and 
therefore allow for customizability without the need for changing the library itself.
\vspace{0.2cm}
\begin{lstlisting}[style=GoStyle, label=changes:function-pointer, caption=An example of a function-pointer addition to the quic-go library.]
    /* Function pointer call within actual quic-go code */
    if packet_setting.ConnectionUpdateBPFHandler != nil /* && potentially other conditions */ {
	    packet_setting.ConnectionUpdateBPFHandler(connId.Bytes(), uint8(connId.Len()), p.connection)
	}
\end{lstlisting}
\vspace{0.2cm}
\begin{lstlisting}[style=GoStyle, label=changes:signature-function-pointer, caption=Only the signature will be defined within the library itself.]
    /* Function pointer signature definition within additional config file */
	ConnectionUpdateBPFHandler      func(id []byte, l uint8, conn QuicConnection) = nil
\end{lstlisting}

The definition of the function that the developer of the relay wished to be executed at the specifically
defined points will be defined locally in the relay code and provided to the configuration of the quic-go library.
An example of how this could look like is shown in \autoref{changes:definition-function-pointer}.
\vspace{0.2cm}
\begin{lstlisting}[style=GoStyle, label=changes:definition-function-pointer, caption=An example of how the addition looks on the relay side.]
    /* Definition of the function within the local relay code */
    func UpdateConnectionId(id []byte, l uint8, conn packet_setting.QuicConnection) {
        /* handle the connection update by interacting with the eBPF-program */
    }   

    /* Providing the function to the quic-go library */
    func main() {
        /* ... */
        packet_setting.ConnectionUpdateBPFHandler = common.UpdateConnectionId
        /* ... */
    }
\end{lstlisting}

The need for these additions arises since the eBPF-program works with its own copy of the current state of a connection.
This, for example, includes the connection-id that will be used when changing the packet header before sending it out.
Since a connection-id can change, i.e.~be updated or retired, during the lifetime of a connection, we need a way to inform 
the eBPF-program to no longer use outdated state-information.
These function-pointer style additions provide a minimal way of adding such functionality without limiting flexibility 
or adding too much application-specific code to the library itself, as it would be the case if the library would access 
the eBPF-Maps directly.

% TODO: mention changes which are not function-pointer style
\subsection{Direct Changes to the Library}
TODO
\section{eBPF Setup}\label{sec:ebpf_setup}
TODO

In order to make the congestion control algorithm that is running in userspace
usable we need to inform the QUIC library about the forwarded packets.
This again happens via BPF maps and a separate go routine that continuously
polls new entries in the map and processes them.
Entries are then added to the packet history to allow the receipt of ACKs.
Besides that, the congestion control algorithm will be informed about the
forwarded packet in order to be able to react to potential congestion events.
\begin{figure}[htbp]
    \centering
    \includegraphics[width=\textwidth]{figures/03_fast_relays/forward-registration.drawio.pdf}
    \caption{Internal setup for registering forwarded packets as well as incorporating forwarding
    limitations for the BPF program.}\label{fig:ebpf-hooks}
\end{figure}

\section{Userspace Synchronization}\label{sec:userspace_synchronization}

Since one of the main ideas we propose is to avoid passing a packet all the way
through the network stack up to userspace at the relay, we create the problem
that the application itself is not aware of packets that are being sent.
To conquer this issue, we suggest a setup that establishes communication
between the application- and the eBPF program.
To keep the improvement we gained by not passing packets to userspace during the 
critical path, this communication will happen in a delayed fashion, that is 
decoupled from the actual sending of media data.

The main resource we use for this communication will again be eBPF maps that
will contain information on the time a packet was sent together with protocol-specific 
data like packet-numbers or stream-identifiers (for both packet-number 
and stream-id the old and the new, i.e.~tranlsated, values will be stored).


% \subsection{User Space Avoidance}
% TODO 

\subsection{Subscription and State Management}
As already mentioned in~\autoref{sec:ebpf_setup}, an eBPF program handling incoming
traffic from the client will save client connection information like MAC address, IP 
address, et cetera, in a map for later access.
Also, an internal counter will give each client a unique identifier. % TODO: mention somewhere that this identifier has to be sequential and therefore does not support "unsubscribing" yet
With that, the only thing that happens in terms of communication between the application 
and the relay in case of a new subscription is that the application will update the ``number 
of clients'' counter that is accessed by the eBPF program and used for packet duplication purposes.

Regarding stream state management, there is also little additional communication since the 
server is expected to use QUIC's unidirectional streams for sending the media data. 
That means the relay does not need to know about the stream details except if it 
has to trigger a retransmission.
% TODO: already mentioned
% If that is the case, the stream-id contained in the packet (meta-)data, that was read from the eBPF map, 
% will be used to manually create a stream with the correct id.
% It is important to manually set the correct id since the relay might not use the same id for the 
% next unidirectional stream it opens.
% Also, the client expects the retransmission to be sent on the same stream-id as the original packet
% since a retransmission happens within the same stream-context.

Since a client can also unsubscribe from a certain media stream, the relay needs to support
this as well.
This is done by simply decrementing the ``number of clients'' counter and making sure the 
client ids stay in a usable state (i.e.~the relay is not duplicating packets for unsubscribed 
clients).
Our prototype implementation does not consider this yet, because our proof of concept and 
performance tests do not require it.

\subsection{Relay Caching}
Caching of data within a relay, which is required by the MoQ standard, is something we essentially get
for free since we are passing on an unaltered copy of any incoming packet from the server to the 
application at the same time we forward all the other packet copies to egress.
This means the application is able to receive any data from the server as if this was a normal connection and 
store, e.g.~the last \verb|n| second(s) worth of data in a cache.
Then the relay could, once a new connection is established, parallel to incrementing the kernel counter 
representing the number of clients also send out cached data already so that the client receives it
as early as possible.

One aspect that we still left open is the point in time when the relay should stop 
sending cached data since the forwarded data is up-to-date.
This also includes the question of how the client can handle potentially duplicate media data if the 
cached- and the forwarded data happen to overlap.
Such questions would likely require further experimenting and testing to find a good solution.

\section{Congestion Considerations}\label{sec:congestion_considerations}
QUIC, like many other modern transport protocols, contains congestion 
control mechanisms regulating the rate at which data is sent to a client.
This is done to avoid the network becoming congested.
Simply forwarding all packets the relay receives from the server would cause 
the relay-client connection to no longer have its own congestion control.
Rather the rate at which the relay sends/forwards to the client would be determined
by the server's congestion control algorithm, i.e.~the network congestion between 
server and relay.
Obviously, this is not a desirable situation, so our approach suggests the eBPF 
program at egress to have its own congestion control functionality.

\subsection{Client Congestion}
Already hinted at in figure~\ref{fig:forward-registration}, it is shown that once a packet is 
registered, there will also be a map update that will be triggered by the congestion controller.
This map update will tell the eBPF egress program how much data it is allowed to send out. 
In figure~\ref{fig:forward-registration} this is visualized exemplary as ``Bytes per Second Limit''
but the idea is that both the function determining how limits and thresholds are calculated from 
the information on incoming packets as well as the actual handling within the egress program 
will be application-specific and defined by the relay engineer.
We experimented with approaches that use the QUIC internal measurements like the RTT, 
introduce new measurements like exponential-weighted moving averages, or even use an 
out-of-band connection where the relay expects direct feedback from the client.
All these possibilities showed us that there is a lot of room for experimentation and optimization
in this area.
This, however, will not be explored further in this thesis and is left for future work.

\subsection{Packet Filtering and Dropping} % TODO: not sure if like this part
Assuming that the network congestion state is known and communicated to the relay, one can 
use the priority-information within a packet (that is expected to be set by the server) to
decide which packets should be forwarded and which ones should be dropped.
One difficulty in this approach is that the dropping mechanism essentially works as an 
online-algorithm, meaning that it does not have full knowledge of the traffic, especially not of 
future packets.
This means that a case like the following could happen:
\begin{itemize}
    \item The traffic contains packets within the priority range of 1 to 5 (5 being the highest priority).
    \item Given the current network congestion to the client, the relay decides to drop 
            all packets below priority 3 and 50\% of the packets with priority 3.
    \item The remaining byte limit to be sent out is running low, and a packet with priority 3 
            comes in, which is sent, since the relay already dropped a lot of previous priority 3 packets.
    \item The next packet turns out to be a priority 5 packet which would overshoot the byte limit if sent.
\end{itemize}
In this example, one could use many different heuristics to handle the situation.
One could always keep enough byte limit left so that a high-priority packet can always be sent.
This, however, essentially just lowers the byte limit for all other packets while making it higher 
for high-priority packets.
Therefore, an individual byte limit per priority could also be used right away.
Also, this approach might cause problems if a lot of high-priority packets come in at once, 
e.g.~in case of very ``bursty'' traffic.
Another way that could be used to handle this uncertain situation is to allow for temporary 
overflows of the byte limit.
This would make the limit more of a soft limit that can be exceeded for a period of short time.
However, this is another heuristic highly dependent on the specific use case and
traffic patterns, so we did not implement it in our prototype.
Overall we can say that implementation-wise, it is not hard to drop packets, but the actual
difficulty lies in finding a reasonable way of deciding which packets to drop.
\section{Integration and Prototype}\label{sec:integration_and_prototype}

In order to show the feasibility of our approach, we have built a prototype 
that implements the suggested setup.
It is capable of streaming an example video as well as mock-payload data from 
a server to a client via a relay.
This setup will be open source and available on GitHub.
Some of the mentioned functionalities like retransmission, caching, and adaptive
bitrate streaming are not fully implemented yet but should not require major
additions since the infrastructure already considers them.
The absence of these functionalities will not have a drastic impact on further 
testing however since the retransmission and the adaptive bitrate streaming will not
be needed in our lab-environment and the caching is not expected to have a big impact
either.

\subsection{Compatibility}
The application layer of the prototype is written in Go but this is not a requirement
itself.
The eBPF program expects QUIC but could also be changed without too much effort to
support any other underlying transport protocol.

\subsection{Source Code Repositories}\label{sec:source_code_repos}
For the development of the relay and the eBPF programs, we have come up with the following repositories:
\begin{itemize}

    \item \textbf{Fast-Relay}~\parencite{adaptive-moq-repo}:
    This is the main repository providing the eBPF program implementations as well as examples of 
    server, relay, and client implementations in Go.
    
    \item \textbf{Quic-Go Adaptation}~\parencite{quic-go-prio-packs-repo}:
    This repository is a fork of the QUIC library ``quic-go''~\parencite{quic-go-repo} and provides a 
    plain Go implementation of the QUIC protocol.
    For our thesis, we needed to make some adaptations to the library to support some hook points for 
    separate functions which should be specifically designed to handle the underlying eBPF setup with its eBPF-Map usage. 
    
    \item \textbf{MoQ-Transport Adaptation}~\parencite{priority-moqtransport-repo}:
    This repository is a fork of the ``MoQ-Transport''~\parencite{draft-moqtransport} protocol repository and provides
    some needed adaptations to our examples. One such adaptation is that the server needs to support a 
    categorization of payloads into different priorities in order for the eBPF program to be able to 
    deliberately drop packets in case of congestion.
    Getting these priorities could be as simple as differentiating only between I- and P-frames in a video 
    stream or more complex based on the needs of the application and the wanted granularity of the congestion 
    control.
    
\end{itemize}
\section{Summary}\label{sec:summary_ch3}

In this chapter, we have presented our approach to designing a relay, that 
is based on eBPF forwarding.
We examined the details of the necessary changes to the underlying QUIC library 
and explored the specific requirements and challenges involved in designing such a system.
All this led us to a detailed explanation, which provides the basis for a 
prototype implementation we developed.
In the upcoming chapter, we will use this prototype to evaluate the 
performance of our relay design compared to a more traditional system.
