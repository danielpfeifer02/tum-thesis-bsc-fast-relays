% !TeX root = ../main.tex
% Add the above to each chapter to make compiling the PDF easier in some editors.

\chapter{Introduction}\label{chap:introduction}

% TODO: does https://arxiv.org/pdf/2310.03256 say cable is not faster than streaming?

The fact that online streaming tends to be slower than cable TV is likely something most people have already 
experienced first-hand.
Live sports events, music shows, or news broadcasts arrive an order of seconds later when streamed compared
to using traditional cable connections. % TODO: citation needed
Despite this delay not necessarily being a deal-breaker for most people, designing networks that 
tighten the gap between cable and streaming is still worthwhile.
With very optimized and fast networks already in place, we are at a point where providing such faster information
delivery is highly non-trivial.
To do that, we have even gone as far as developing entirely new standards, such as QUIC\@. 
Those new standards aim to improve the shortcomings of some of the most fundamental protocols of the Internet, one of them being TCP, which have been around more than 40 years. 
\\
Besides introducing new protocols, one could also look at existing setups and figure out how to trade some generality
for a smaller delay when handling data.
The ISO/OSI model, which is a foundational concept in networking, provides ``a common basis for the coordination 
of standards development for the purpose of systems interconnection''~\parencite{iso-osi-standard}.
As one can imagine, such a ``common basis'', even though convenient for large-scale systems, can cause unnecessary
overhead.
In some cases, additional speed-ups can be achieved by using more application-specific approaches.
This thesis will consider one such case and explore the possibilities of avoiding or delaying specific processing
steps of the ISO/OSI model to increase the overall speed of a network relay.

\section{Research Question}\label{sec:research_question}

As already mentioned above, the usage of application-specific approaches in networking allows for a reduction in latency.
In this thesis, we will consider a media streaming scenario that runs on top of QUIC by using the ``Media over QUIC'' (MoQ) transport protocol
~\parencite{draft-moqtransport}.
The central question we will try to answer in this thesis will be:
\vspace{0.5cm}
\begin{center}
    \textit{How can we improve the performance of a relay in a media streaming scenario by using eBPF technology?}
\end{center}
\vspace{0.5cm}
By using eBPF technology together with kernel hook points provided by the Linux-Kernel, we will try to find a setup that improves relay 
performance using eBPF programs that handle basic relay capabilities, such as packet forwarding and congestion control.
Since the QUIC protocol is designed to handle a significant portion of its workload in userspace, we look into possibilities of delaying any 
userspace processing until \textbf{after} the packet has been forwarded to the client.
This way, the raw delay that the packet experiences from the initial media server to the client can be reduced. 
However, since QUIC is a connection-oriented protocol, we need to make sure that the QUIC connection state stays 
coherent despite the additional processing steps done by the eBPF program.
We will investigate which additional processing steps are needed in our case, how they compare to challenges when expanding our approach to other protocols, 
and how they can be implemented using eBPF\@.
Therefore, more specific sub-questions we try to answer are:
\vspace{0.5cm}
\begin{enumerate}
    % \item \textit{How can we avoid the need to direct a packet through userspace?}
    \item \textit{How can we remove userspace packet-processing from the critical path?}
    % \item \textit{How to handle the fact that packets are heavily encrypted?}
    \item \textit{How to handle heavy packet encryption?}
    \item \textit{What communication between userspace and eBPF program is necessary to stay coherent?}
    \item \textit{How can our approach be generalized to other protocols?}
\end{enumerate}
\vspace{0.5cm}
\section{Scope}\label{sec:scope}

The main improvement this thesis aims to achieve is shortening the critical path a packet takes from a media server to a client.
This will be done by avoiding the immediate need of a packet traversal up the network stack to the application layer.
Instead, any communication with the application layer will happen in a delayed fashion (after the packet was sent) by utilizing 
eBPF-Maps for storing any necessary (meta-)~information.
The main reason this communication between userspace and eBPF-program is required lies in the fact that relays in MoQ are an application layer concept.
That means the QUIC connections to from relay to server and from relay to client will be different and the packets that have been eBPF-forwarded to 
egress directly will need changes in their header data in order to match the state of the outgoing client connection.  
\\
This approach is highly dependent on the used standards and protocols.
This thesis will operate on top of the QUIC protocol~\parencite{rfc-9000} and the ``Media over QUIC'' (MoQ) 
transport protocol~\parencite{draft-moqtransport}.
For the application layer the quic-go library~\parencite{quic-go-repo} will provide the implementation and 
any additional (non-eBPF) program will also be written in Go.
Since the setup is dependent on retreiving data from eBPF-Maps the QUIC library providing the implementation 
will need some adaptations.
We will mainly introduce simple function pointer style additions that allow the adapted library to be run 
both with and without the eBPF setup.
The developer of the relay will then also have more freedom to setup the eBPF part of the relay as they see fit
since the Go code that will interact with eBPF parts will also have to be provided by said developer.
\\
Additionally we will run a performance analysis on our implementation of the relay to confirm the potential this 
approach has.
These performance tests will look at the raw delay speedup as well as the impact on CPU utilization this 
setup has.
All the tests will be done in a lab-like environment to isolate the performance changes as best as possible
from any outside noise.
The payloads used will only contain dummy data since our approach does not interfer with payload contents 
and there is no need for creating and using real media stream data.
\\
Despite our approach only considering QUIC and MoQ, we will argue that the general idea of our setup will be independent of
any of these protocols and can be changed to fit ones needs.
\\
With this we will provide answers to the research questions regarding packet-redirection, communication between userspace and eBPF
as well as setup-generalization.
Regarding the question on how to handle the encryption of the packets, we will not focus on this since we did not find a suitable
hardware offload that would have allow for en- and decryption after and before the used eBPF hook points respectively.
Instead we will emulate this behavior by turning off the encryption in the QUIC library itself which will provide a similar result.

% TODO: better spot for this part?
% The only two assumptions that were made are that it is possible for the protocol to store meta-information like a 
% packet-priority in the packet itself and that our relay has access header data even if it is part of encrypted areas.
% For the former point we utilized one byte of the QUIC Connection-Id as a proof of conecept but in later implementations 
% this could be changed to using any separately defined application-header that allows for the storage of such 
% information.
% Since we assume full knowledge over the eBPF setup when the payload is created at the server (because server and 
% relay are generally both part of the same CDN) this assumption is not a big stretch.
% The latter point can be achieved by having a hardware offload of en- and decryption onto a smartNIC\@.
% Since such an offload was not yet available to us we turned of en- and decryption directly within the 
% QUIC library itself which provides a similar result.

\section{Structure of this Thesis}\label{sec:structure_of_thesis}

In chapter\nobreakspace\ref{chap:background}, this thesis will provide some overview of used technologies
and related ideas.
Section~\ref{sec:quic_bg} will give an introduction to the QUIC protocol and its main features and 
section\nobreakspace\ref{sec:ebpf_bg} will provide an overview of eBPF technology together with features related to 
our approach.
Section~\ref{sec:moq_bg} will introduce the `Media over QUIC' (MoQ) transport protocol which will be used 
for our application-level relay setup.
After that section\nobreakspace\ref{sec:rt_and_adaptive_bitrate_streaming} will explain the ideas and challenges of 
real time streaming as well as adaptive bitrate streaming while section\nobreakspace\ref{sec:related_work} will 
mention some work related to the aforementioned topics.
What will follow in chapter\nobreakspace\ref{chap:fast_relays} is a detailed description of the setup that 
allowed us to improve relay performance. 
We will look at the adaptations to the used QUIC library in section\nobreakspace\ref{sec:quic_adaptions} as well as 
our eBPF setup in~\ref{sec:ebpf_setup}. Besides those two, we will also look at some more specific details and 
challenges in the subsequent sections.
In chapter\nobreakspace\ref{chap:testing}, we will then provide a basic performance analysis of our setup to show current improvements and limitations.
Finally, we will conclude with a summary together with some ideas for future work in this field in chapters\nobreakspace\ref{chap:future_work} 
and\nobreakspace\ref{chap:conclusion}.


% % TODO: remove %
% \section{Citation Examples}
% Citation~\parencite{rfc-9000}.
% Citation~\parencite{iso-osi-standard}.
% Citation~\parencite{draft-moqtransport}.
% Citation~\parencite{article-quic-usage}.
% Citation~\parencite{internet-quic-usage}.
% Citation~\parencite{facebook-quic-usage}.
% Citation~\parencite{google-quic-usage}.
% Citation~\parencite{quic-nic-offload}.
% Citation~\parencite{quic-explained}.
% Citation~\parencite{equic-gateway}.
% Citation~\parencite{media-streaming-prio-drop}.
% Citation~\parencite{quic-nic-offload-patent}.
% Citation~\parencite{kernel-bypass-msc-thesis}.
% Citation~\parencite{quic-go-repo}.
% Citation~\parencite{quic-go-prio-packs-repo}.
% Citation~\parencite{adaptive-moq-repo}.
% Citation~\parencite{priority-moqtransport-repo}.
% Citation~\parencite{fast-relays-thesis-repo}.
% Citation~\parencite{ebpf-verifier}.